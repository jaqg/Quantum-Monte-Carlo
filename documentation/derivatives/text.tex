\graphicspath{{./figures/}}

% ------
\section{Definitions}
% ------
The molecular orbitals (MO's), our wavefunctions, $\psi$, are formed as Linear
Combinations of atomic orbitals (AO's).
The AO basis functions used are the normalized $1s$ \textit{Slater type function} 
centered at $\mat{R}$
\begin{equation}
    \phi \left( \mat{r} - \mat{R} \right) \equiv
    \phi \left( \mat{r} \right) =
    \left( \frac{\alpha^3}{\pi} \right)^{1 /2}
    e^{-\alpha \left| \mat{r} - \mat{R} \right|}
    ,
\end{equation}
where $\alpha$ is the \textit{Slater orbital exponent} and 
$\left| \mat{r} - \mat{R} \right|$ is the distance between a point in space,
$ \mat{r}$ and $ \mat{R}$ 
\begin{equation} \label{eq:definition-mod-rR}
    \left| \mat{r} - \mat{R} \right|
    =
    \sqrt{\left( x - X \right)^{2} + \left( y - Y \right)^{2} + \left( z - Z \right)^{2}}
    ,
\end{equation}
where lower case letters reffer to the cartesian coordinates of 
$ \mat{r} = \left( x\ y\ z \right)$ and upper case to 
$ \mat{R} = \left( X\ Y\ Z \right)$.

To simplify the notation, the value of the orbital in the $i$-th point in space,
$\mat{r}_i = \left( x_i\ y_i\ z_i \right)$, centered on the $j$-th atom, at
$\mat{R}_j = \left( X_j\ Y_j\ Z_j \right)$, is denoted as 
\begin{equation}
    \phi_{ij} \equiv
    \phi \left( \mat{r}_i - \mat{R}_j \right) =
    \left( \frac{\alpha^3}{\pi} \right)^{1 /2}
    e^{-\alpha \left| \mat{r}_i - \mat{R}_j \right|}
    \equiv
    \left( \frac{\alpha^3}{\pi} \right)^{1 /2}
    e^{-\alpha \left| \mat{r}_{ij} \right|}
    ,
\end{equation}
where it is important to note that the modulus $\left| \mat{r}_{i} - \mat{R}_{j} \right|$
is simply written as the two-indexed $\left| \mat{r}_{ij} \right|$.

Therefore, our MO's are constructed in the AO's basis functions $\left\{ \phi_{\mu} \right\}$
as the product of the summed contributions of all the AO's at any given point
in space. More specifically, as the product of the summed contributions of each
orbital, centered at each atom position, for the position of each electron.
Then, the wavefunction for $n$ electrons and $m$ atoms (nucleus) is given by 
\begin{equation} \label{eq:definition-psi}
    \psi \left( \mat{r} \right) =
    \prod_{i=1}^{n} \left( \sum_{I=1}^{m} \phi_{iI} \right)
    ,
\end{equation}
where, to keep the notation simple, the electrons are indexed with lower case
letters and the nucleus with upper case.
For $n$ electrons and $m$ nucleus, $\mat{r}$ has $3n$ components and
$\mat{R}$ has $3m$ components.

The drif vector is given by 
\begin{equation} \label{eq:definition-drift-vector}
    \frac{
        \nabla \psi \left( \mat{r} \right)
    }{
        \psi \left( \mat{r} \right)
    }
    ,
\end{equation}
where the gradient $\nabla \psi \left( \mat{r} \right)$ reads
\begin{equation} \label{eq:definition-gradient}
    \nabla
    \psi \left( \mat{r} \right)
    =
    \nabla
    \psi \left( x_1, y_1, z_1, \ldots, x_n, y_n, z_n \right)
    =
    \begin{pmatrix}
        \partial_{x_1} \psi \\
        \partial_{y_1} \psi \\
        \partial_{z_1} \psi \\
        \vdots \\
        \partial_{x_n} \psi \\
        \partial_{y_n} \psi \\
        \partial_{z_n} \psi
    \end{pmatrix}
    .
\end{equation}

The local kinetic energy is defined as 
\begin{equation} \label{eq:definition-kinetic-energy}
    T_{\mathrm{L}} \left( \mat{r} \right)
    =
    - \frac{1}{2} \frac{\laplacian \psi \left( \mat{r} \right)}{\psi \left( \mat{r} \right)}
    ,
\end{equation}
where the laplacian reads
\begin{equation} \label{eq:definition-laplacian}
    \laplacian 
    =
    \partial_x^{2} + \partial_y^{2} + \partial_z^{2}
    .
\end{equation}

To compute the drif vector and kinetic energy, the gradient and laplacian of
the wavefunction over the electron positions, $ \mat{r}$, are needed.
Firstly, to easy the upcoming development, the first and second partial 
derivatives of the basis functions are calculated.

% ------
\section{Partial derivatives of the AO's}
% ------
The first partial derivative of the AO
$\phi \left( \mat{r}_i - \mat{R}_I \right) \equiv \phi_{iI}$
over the $x$ component of the $i$-th
electron, $x_i$, is given by 
\begin{equation}
    \frac{\partial}{\partial x_i} \phi_{iI}
    \equiv
    \partial_{x_i} \phi_{iI}
    .
\end{equation}

By definition
\begin{equation}
    \partial_{x_{i}} f\left( \mat{r}_i \right)
    =
    \partial_{\mat{r}_{i}} f\left( \mat{r}_i \right) \partial_{x_{i}} \mat{r}_{i}
    ,
\end{equation}
then 
\begin{equation} \label{eq:fd-phi-1}
    \partial_{x_{i}} \phi_{iI}
    =
    \partial_{\left| \mat{r}_{iI} \right|} \phi_{iI} \partial_{x_{i}} \left| \mat{r}_{iI} \right|
    ,
\end{equation}
where the first partial derivative reads 
\begin{equation}
    \partial_{\left| \mat{r}_{iI} \right|} \phi_{iI}
    =
    \left( \frac{\alpha^3}{\pi} \right)^{1 /2}
    \partial_{\left| \mat{r}_{iI} \right|} 
    e^{-\alpha \left| \mat{r}_{iI} \right|}
    =
    -\alpha
    \left( \frac{\alpha^3}{\pi} \right)^{1 /2}
    e^{-\alpha \left| \mat{r}_{iI} \right|}
    =
    -\alpha \phi_{iI}
    ,
\end{equation}
and the second one reads 
\begin{align} \label{eq:partial-xi-riI}
    \notag
    \partial_{x_{i}} \left| \mat{r}_{iI} \right|
    &=
    \partial_{x_{i}} \left[ \left( x_i - X_I \right)^{2} + \left( y_i - Y_I \right)^{2} + \left( z_i - Z_I \right)^{2}  \right]^{1 /2}
    \\
    \notag
    &=
    \frac{1}{2} \partial_{x_{i}} \left[ \left( x_i - X_I \right)^{2} + \left( y_i - Y_I \right)^{2} + \left( z_i - Z_I \right)^{2}  \right] \left[ \left( x_i - X_I \right)^{2} + \left( y_i - Y_I \right)^{2} + \left( z_i - Z_I \right)^{2}  \right]^{-1 /2}
    \\
    \notag
    &=
    \frac{1}{2} 2 \partial_{x_{i}} \left( x_i - X_I \right) \left( x_i - X_I \right) \frac{1}{\left| \mat{r}_{iI} \right|}
    \\
    &=
    \frac{x_i - X_I}{\left| \mat{r}_{iI} \right|}
    .
\end{align}

Then, in \cref{eq:fd-phi-1}
\begin{equation} \label{eq:partial-phi-xi}
    \partial_{x_{i}} \phi_{iI}
    =
    -\alpha
    \frac{x_i - X_I}{\left| \mat{r}_{iI} \right|}
    \phi_{iI} 
    .
\end{equation}
For the $y_i$ and $z_i$ components, the analogous result is obtained.
Therefore, the first partial derivative over the 
$\lambda_i = \left\{ x_i, y_i, z_i \right\}$
component is given by
\begin{equation} \label{eq:partial-phi-lambda}
    \therefore
    \partial_{\lambda_{i}} \phi_{iI}
    =
    -\alpha
    C_{\lambda,iI}
    \phi_{iI}
    ,
\end{equation}
where 
\begin{equation} \label{eq:definition-CiI}
    C_{\lambda,iI} =
    \frac{\lambda_i - \Lambda_I}{\left| \mat{r}_{iI} \right|}
    .
\end{equation}

The second partial derivative is obtained as
\begin{equation} \label{eq:der2-xi-phi-1}
    \partial_{x_{i}}^{2} \phi_{iI}
    =
    -\alpha
    \partial_{x_{i}}
    \left( 
        C_{x,iI}
        \phi_{iI} 
    \right)
    .
\end{equation}

Applying the product rule
\begin{equation} \label{eq:der2-xi-phi-2}
    \partial_{x_i}
    \left( 
        C_{x,iI}
        \phi_{iI} 
    \right)
    =
    \phi_{iI}
    \partial_{x_i}
    C_{x,iI}
    +
    C_{x,iI}
    \partial_{x_i}
    \phi_{iI}
    ,
\end{equation}
where, by the quotient rule
\begin{equation}
    \partial_{x_i}
    C_{x,iI}
    =
    \partial_{x_i}
    \left( 
        \frac{x_i - X_I}{\left| \mat{r}_{iI} \right|}
    \right)
    =
    \frac{
        \left| \mat{r}_{iI} \right|
        \partial_{x_i}
        \left( x_i - X_I \right)
        %
        -
        %
        \left( x_i - X_I \right)
        \partial_{x_i}
        \left| \mat{r}_{iI} \right|
    }{
        \left| \mat{r}_{iI} \right|^{2}
    }
    ,
\end{equation}
as $\partial_{x_i} \left( x_i - X_I \right) = 1$ and recalling \cref{eq:partial-xi-riI}
\begin{equation}
    \partial_{x_i}
    C_{x,iI}
    =
    \frac{
        \left| \mat{r}_{iI} \right|
        %
        -
        %
        \left( x_i - X_I \right)
        \frac{x_i - X_I}{\left| \mat{r}_{iI} \right|}
    }{
        \left| \mat{r}_{iI} \right|^{2}
    }
    =
    \frac{
        1
    }{
        \left| \mat{r}_{iI} \right|
    }
    -
    \frac{
        \left( x_i - X_I \right)^{2}
    }{
        \left| \mat{r}_{iI} \right|^{3}
    }
    =
    \frac{
        1 - C_{x,iI}^{2}
    }{
        \left| \mat{r}_{iI} \right|
    }
    ,
\end{equation}

In \cref{eq:der2-xi-phi-2}
\begin{equation} \label{eq:partial-xiXI-riI-1}
    \partial_{x_i}
    \left(
        C_{x,iI}
        \phi_{iI}
    \right)
    =
    \frac{
        1 - C_{x,iI}^{2}
    }{
        \left| \mat{r}_{iI} \right|
    }
    \phi_{iI}
    +
    C_{x,iI}
    \partial_{x_i}
    \phi_{iI}
    ,
\end{equation}
and in \cref{eq:der2-xi-phi-1} 
\begin{equation}
    \partial_{x_{i}}^{2} \phi_{iI}
    =
    -\alpha
    \left( 
        \frac{
            1 - C_{x,iI}^{2}
        }{
            \left| \mat{r}_{iI} \right|
        }
        \phi_{iI}
        +
        C_{x,iI}
        \partial_{x_i}
        \phi_{iI}
    \right)
    ,
\end{equation}
inserting the $-\alpha$ factor 
\begin{equation} \label{eq:der2-xi-phi-3}
    \partial_{x_{i}}^{2} \phi_{iI}
    =
    -
    \frac{
        \alpha
    }{
        \left| \mat{r}_{iI} \right|
    }
    \phi_{iI}
    -
    \left( 
        -\alpha
        \frac{
            C_{x,iI}^{2}
        }{
            \left| \mat{r}_{iI} \right|
        }
        \phi_{iI}
    \right)
    - \alpha
    C_{x,iI}
    \partial_{x_i}
    \phi_{iI}
    ,
\end{equation}
and recalling \cref{eq:partial-phi-lambda}, the second term reads
\begin{equation}
    -\alpha
    \frac{
        C_{x,iI}^{2}
    }{
        \left| \mat{r}_{iI} \right|
    }
    \phi_{iI}
    =
    \frac{
        C_{x,iI}
    }{
        \left| \mat{r}_{iI} \right|
    }
    \left(
        - \alpha
        C_{x,iI}
        \phi_{iI}
    \right)
    =
    \frac{
        C_{x,iI}
    }{
        \left| \mat{r}_{iI} \right|
    }
    \partial_{x_{i}} \phi_{iI}
    .
\end{equation}

In \cref{eq:der2-xi-phi-3}
\begin{equation}
    \partial_{x_{i}}^{2} \phi_{iI}
    =
    -
    \frac{
        \alpha
    }{
        \left| \mat{r}_{iI} \right|
    }
    \phi_{iI}
    -
    \frac{
        C_{x,iI}
    }{
        \left| \mat{r}_{iI} \right|
    }
    \partial_{x_{i}} \phi_{iI}
    - \alpha
    C_{x,iI}
    \partial_{x_i}
    \phi_{iI}
    ,
\end{equation}
and taking $\partial_{x_i} \phi_{iI}$ as common factor
\begin{equation}
    \partial_{x_{i}}^{2} \phi_{iI}
    =
    -
    \frac{
        \alpha
    }{
        \left| \mat{r}_{iI} \right|
    }
    \phi_{iI}
    -
    \left( 
        \frac{
            C_{x,iI}
        }{
            \left| \mat{r}_{iI} \right|
        }
        +
        \alpha
        C_{x,iI}
    \right)
    \partial_{x_i}
    \phi_{iI}
    ,
\end{equation}
or
\begin{equation}
    \partial_{x_{i}}^{2} \phi_{iI}
    =
    -
    \frac{
        \alpha
    }{
        \left| \mat{r}_{iI} \right|
    }
    \phi_{iI}
    -
    \frac{
        C_{x,iI}
        \left( 
            1 + \alpha \left| \mat{r}_{iI} \right|
        \right)
    }{
        \left| \mat{r}_{iI} \right|
    }
    \partial_{x_i}
    \phi_{iI}
    .
\end{equation}

The second partial derivative over the 
$\lambda_i = \left\{ x_i, y_i, z_i \right\}$
component can be expressed in terms of the first partial derivative as
\begin{equation} \label{eq:partial-2-phi-lambda-first-der}
    \therefore
    \partial_{\lambda_{i}}^{2} \phi_{iI}
    =
    -
    \frac{
        \alpha
    }{
        \left| \mat{r}_{iI} \right|
    }
    \phi_{iI}
    -
    \frac{
        C_{\lambda,iI}
        \left( 
            1 + \alpha \left| \mat{r}_{iI} \right|
        \right)
    }{
        \left| \mat{r}_{iI} \right|
    }
    \partial_{\lambda_i}
    \phi_{iI}
    ,
\end{equation}
which can be usefull for testing.

Now, recalling \cref{eq:partial-phi-lambda}
\begin{equation}
    \partial_{x_{i}}^{2} \phi_{iI}
    =
    -
    \frac{
        \alpha
    }{
        \left| \mat{r}_{iI} \right|
    }
    \phi_{iI}
    -
    \frac{
        C_{x,iI}
        \left( 
            1 + \alpha \left| \mat{r}_{iI} \right|
        \right)
    }{
        \left| \mat{r}_{iI} \right|
    }
    \left( 
        -\alpha
        C_{\lambda,iI}
        \phi_{iI}
    \right)
    ,
\end{equation}
and taking $-\alpha \phi_{iI}$ as common factor
\begin{equation}
    \partial_{x_{i}}^{2} \phi_{iI}
    =
    - \alpha
    \left( 
        \frac{
            1
        }{
            \left| \mat{r}_{iI} \right|
        }
        -
        \frac{
            C_{x,iI}^{2}
            \left( 
                1 + \alpha \left| \mat{r}_{iI} \right|
            \right)
        }{
            \left| \mat{r}_{iI} \right|
        }
    \right)
    \phi_{iI}
    .
\end{equation}
the second partial derivative over the 
$\lambda_i = \left\{ x_i, y_i, z_i \right\}$
component is given by
\begin{equation} \label{eq:partial-2-phi-lambda}
    \therefore
    \partial_{\lambda_{i}}^{2} \phi_{iI}
    =
    -\alpha
    D_{\lambda,iI}
    \phi_{iI}
    ,
\end{equation}
where 
\begin{equation} \label{eq:coeficiente-DiI-CiI}
    D_{\lambda,iI}
    =
    \frac{
        1
    }{
        \left| \mat{r}_{iI} \right|
    }
    -
    \frac{
        C_{\lambda,iI}^{2}
        \left( 
            1 + \alpha \left| \mat{r}_{iI} \right|
        \right)
    }{
        \left| \mat{r}_{iI} \right|
    }
    ,
\end{equation}
or, recalling the definition of $C_{\lambda,iI}$ in \cref{eq:definition-CiI}
\begin{equation} \label{eq:coeficiente-DiI}
    D_{\lambda,iI} =
    \frac{1}{\left| \mat{r}_{iI} \right|}
    -
    \frac{
        \left( \lambda_i - \Lambda_I \right)^{2}
        \left(  
            1 + \alpha \left| \mat{r}_{iI} \right|
        \right)
    }{
        \left| \mat{r}_{iI} \right|^{3}
    }
    .
\end{equation}

% ------
\section{Partial derivatives of the wavefunction}
% ------
The first derivative of the wavefunction
$\psi \left( \mat{r} \right) \equiv \psi$
over the $x$ component of the $i$-th
electron, $x_i$, is given by 
\begin{equation}
    \partial_{x_i} \psi
    =
    \partial_{x_i} \prod_{i=1}^{n} \left( \sum_{I=1}^{m} \phi_{iI} \right)
    .
\end{equation}

As the partial derivative of all the $j \not= i$ components over $i$ is null 
\begin{equation}
    \partial_{x_j} \phi_{jI} = 0\quad \forall j \not= i\ \mathrm{and}\ \forall I,
\end{equation}
only the $j\not= i$ components that are not differentiated survive. Then 
\begin{equation} \label{eq:fd-psi-1}
    \partial_{x_i} \psi 
    =
    \partial_{x_i} \prod_{i=1}^{n} \left( \sum_{I=1}^{m} \phi_{iI} \right)
    =
    \prod_{j\not=i}^{n} \left( \sum_{I=1}^{m} \phi_{jI} \right)
    \partial_{x_i} \left( \sum_{I=1}^{m} \phi_{iI} \right)
    =
    \prod_{j\not=i}^{n} \left( \sum_{I=1}^{m} \phi_{jI} \right)
    \sum_{I=1}^{m} \partial_{x_i} \phi_{iI}
    .
\end{equation}

Since both in the drift vector and the kinetic energy 
(\cref{eq:definition-drift-vector,eq:definition-kinetic-energy}) the gradient and the laplacian
are divided by $\psi$, it is convinient to express those in terms of
$\psi$.

Given the definition of $\psi$ in \cref{eq:definition-psi}, the product can be
separated as
\begin{equation}
    \psi =
    \left( \sum_{I=1}^{m} \phi_{iI} \right)
    \prod_{j\not=i}^{n} \left( \sum_{I=1}^{m} \phi_{jI} \right)
    ,
\end{equation}
which implies that
\begin{equation}
    \prod_{j\not=i}^{n} \left( \sum_{I=1}^{m} \phi_{jI} \right)
    =
    \frac{\psi}{
        \sum_{I=1}^{m} \phi_{iI}
    }
    .
\end{equation}

This way, \cref{eq:fd-psi-1} can be rewritten in terms of $\psi$ and 
$\partial_{x_i} \phi_{iI}$ as
\begin{equation} \label{eq:fd-psi-2}
    \partial_{x_i} \psi 
    =
    \frac{
        \sum_{I=1}^{m} \partial_{x_i} \phi_{iI}
    }{
        \sum_{I=1}^{m} \phi_{iI}
    }
    \psi
    .
\end{equation}

Recalling \cref{eq:partial-phi-lambda} in \cref{eq:fd-psi-2}
\begin{equation} \label{eq:derivada-psi-xi}
    \partial_{x_i} \psi 
    =
    -\alpha
    \frac{
        \sum_{I=1}^{m} 
        C_{x,iI}
        \phi_{iI} 
    }{
        \sum_{I=1}^{m} \phi_{iI}
    }
    \psi
    .
\end{equation}

Therefore, the first partial derivative over the 
$\lambda_i = \left\{ x_i, y_i, z_i \right\}$
components reads
\begin{equation} \label{eq:partial-psi-lambda}
    \therefore
    \partial_{\lambda_i} \psi 
    =
    -\alpha
    \frac{
        \sum_{I=1}^{m} 
        C_{\lambda,iI}
        \phi_{iI} 
    }{
        \sum_{I=1}^{m} \phi_{iI}
    }
    \psi
    ,
\end{equation}
where $C_{\lambda,iI}$ is given by \cref{eq:definition-CiI}.

For the laplacian, the second partial derivatives are needed.
Following the same scheme, the partial derivative over the $x_i$ component of
$\partial_{x_i} \psi$ from  \cref{eq:fd-psi-2} is taken 
\begin{equation}
    \partial_{x_i}^{2} \psi =
    \partial_{x_i}
    \left(
        \frac{
            \sum_{I=1}^{m} \partial_{x_i} \phi_{iI}
        }{
            \sum_{I=1}^{m} \phi_{iI}
        }
        \psi
    \right)
    ,
\end{equation}
that, by the product rule
\begin{equation} \label{eq:sd-psi-1}
    \partial_{x_i}^{2} \psi =
    \psi
    \underbrace{
        \partial_{x_i}
        \left( 
            \frac{
                \sum_{I=1}^{m} \partial_{x_i} \phi_{iI}
            }{
                \sum_{I=1}^{m} \phi_{iI}
            }
        \right)
    }_{\left( \mathrm{a} \right)}
    +
    \underbrace{
        \frac{
            \sum_{I=1}^{m} \partial_{x_i} \phi_{iI}
        }{
            \sum_{I=1}^{m} \phi_{iI}
        }
        \partial_{x_i}
        \psi
    }_{\left( \mathrm{b} \right)}
    .
\end{equation}

The first partial derivative is treated separately.
Applying the quotient rule
\begin{equation} \label{eq:sd-psi-2}
    \left( \mathrm{a} \right)\
    \partial_{x_i}
    \left( 
        \frac{
            \sum_{I=1}^{m} \partial_{x_i} \phi_{iI}
        }{
            \sum_{I=1}^{m} \phi_{iI}
        }
    \right)
    =
    \frac{
        \left( 
            \sum_{I=1}^{m} \phi_{iI}
        \right)
        \partial_{x_i}
        \left( 
            \sum_{I=1}^{m} \partial_{x_i} \phi_{iI}
        \right)
    }{
        \left( 
            \sum_{I=1}^{m} \phi_{iI}
        \right)^{2}
    }
    %
    -
    %
    \frac{
        \left( 
            \sum_{I=1}^{m} \partial_{x_i} \phi_{iI}
        \right)
        \partial_{x_i}
        \left( 
            \sum_{I=1}^{m} \phi_{iI}
        \right)
    }{
        \left( 
            \sum_{I=1}^{m} \phi_{iI}
        \right)^{2}
    }
    ,
\end{equation}
which simplifies to
\begin{equation} \label{eq:sd-psi-3}
    \left( \mathrm{a} \right)\
    \partial_{x_i}
    \left( 
        \frac{
            \sum_{I=1}^{m} \partial_{x_i} \phi_{iI}
        }{
            \sum_{I=1}^{m} \phi_{iI}
        }
    \right)
    =
    \underbrace{
        \frac{
            \sum_{I=1}^{m} \partial_{x_i}^{2} \phi_{iI}
        }{
            \sum_{I=1}^{m} \phi_{iI}
        }
    }_{\left( \mathrm{c} \right)}
    %
    -
    %
    \frac{
        \left( 
            \sum_{I=1}^{m} \partial_{x_i} \phi_{iI}
        \right)^{2}
    }{
        \left( 
            \sum_{I=1}^{m} \phi_{iI}
        \right)^{2}
    }
    .
\end{equation}

Again, the first term is calculated separately.
Recalling \cref{eq:partial-2-phi-lambda}
\begin{equation}
    \left( \mathrm{c} \right)\ 
    \frac{
        \sum_{I=1}^{m} \partial_{x_i}^{2} \phi_{iI}
    }{
        \sum_{I=1}^{m} \phi_{iI}
    }
    =
    -\alpha
    \frac{
        \sum_{I=1}^{m}
        D_{x,iI}
        \phi_{iI}
    }{
        \sum_{I=1}^{m} \phi_{iI}
    }
    ,
\end{equation}
in \cref{eq:sd-psi-3}
\begin{equation} \label{eq:derivada-a}
    \left( \mathrm{a} \right)\
    \partial_{x_i}
    \left( 
        \frac{
            \sum_{I=1}^{m} \partial_{x_i} \phi_{iI}
        }{
            \sum_{I=1}^{m} \phi_{iI}
        }
    \right)
    =
    -\alpha
    \frac{
        \sum_{I=1}^{m}
        D_{x,iI}
        \phi_{iI}
    }{
        \sum_{I=1}^{m} \phi_{iI}
    }
    %
    -
    %
    \left( 
        \frac{
            \sum_{I=1}^{m}
            \partial_{x_i}
            \phi_{iI}
        }{
            \sum_{I=1}^{m} \phi_{iI}
        }
    \right)^{2}
    .
\end{equation}

The second term in \cref{eq:sd-psi-1}, recalling \cref{eq:fd-psi-2}, reads
\begin{equation} \label{eq:derivada-b}
    \left( \mathrm{b} \right)\
    \frac{
        \sum_{I=1}^{m} \partial_{x_i} \phi_{iI}
    }{
        \sum_{I=1}^{m} \phi_{iI}
    }
    \partial_{x_i}
    \psi
    =
    \frac{
        \sum_{I=1}^{m} \partial_{x_i} \phi_{iI}
    }{
        \sum_{I=1}^{m} \phi_{iI}
    }
    \frac{
        \sum_{I=1}^{m} \partial_{x_i} \phi_{iI}
    }{
        \sum_{I=1}^{m} \phi_{iI}
    }
    \psi
    =
    \left( 
        \frac{
            \sum_{I=1}^{m} \partial_{x_i} \phi_{iI}
        }{
            \sum_{I=1}^{m} \phi_{iI}
        }
    \right)^{2}
    \psi
    .
\end{equation}

Substituting \cref{eq:derivada-a,eq:derivada-b} into \cref{eq:sd-psi-1}
\begin{equation}
    \partial_{x_i}^{2} \psi =
    \left[ 
        -\alpha
        \frac{
            \sum_{I=1}^{m}
            D_{\lambda,iI}
            \phi_{iI}
        }{
            \sum_{I=1}^{m} \phi_{iI}
        }
        %
        -
        %
        \left( 
            \frac{
                \sum_{I=1}^{m}
                \partial_{x_i}
                \phi_{iI}
            }{
                \sum_{I=1}^{m} \phi_{iI}
            }
        \right)^{2}
    \right]
    \psi
    +
    \left( 
        \frac{
            \sum_{I=1}^{m} \partial_{x_i} \phi_{iI}
        }{
            \sum_{I=1}^{m} \phi_{iI}
        }
    \right)^{2}
    \psi
    ,
\end{equation}
which reduces to
\begin{equation}
    \partial_{x_i}^{2} \psi =
    -\alpha
    \frac{
        \sum_{I=1}^{m}
        D_{x,iI}
        \phi_{iI}
    }{
        \sum_{I=1}^{m} \phi_{iI}
    }
    \psi
    .
\end{equation}
Therefore, the second partial derivative over the 
$\lambda_i = \left\{ x_i, y_i, z_i \right\}$
components reads
\begin{equation} \label{eq:derivada-psi2-lambda}
    \therefore
    \partial_{\lambda_i}^{2} \psi =
    -\alpha
    \frac{
        \sum_{I=1}^{m}
            D_{\lambda,iI}
        \phi_{iI}
    }{
        \sum_{I=1}^{m} \phi_{iI}
    }
    \psi
    ,
\end{equation}
where $D_{\lambda,iI}$ is given by \cref{eq:coeficiente-DiI}.

% ------
\section{Gradient \& drift vector}
% ------
Substituting \cref{eq:partial-psi-lambda}
into the definition of the gradient given in \cref{eq:definition-gradient}
\begin{equation}
    \nabla
    \psi \left( \mat{r} \right)
    =
    \begin{pmatrix}
        \partial_{x_1} \psi \\
        \partial_{y_1} \psi \\
        \partial_{z_1} \psi \\
        \vdots \\
        \partial_{x_n} \psi \\
        \partial_{y_n} \psi \\
        \partial_{z_n} \psi
    \end{pmatrix}
    =
    \begin{pmatrix}
        -\alpha
        \left( 
            \sum_{I=1}^{m} 
            C_{x,1I}
            \phi_{1I} 
        \right)
        \left( 
            \sum_{I=1}^{m} \phi_{1I}
        \right)^{-1}
        \psi
        \\
        %
        -\alpha
        \left( 
            \sum_{I=1}^{m} 
            C_{y,1I}
            \phi_{1I} 
        \right)
        \left( 
            \sum_{I=1}^{m} \phi_{1I}
        \right)^{-1}
        \psi
        \\
        %
        -\alpha
        \left( 
            \sum_{I=1}^{m} 
            C_{z,1I}
            \phi_{1I} 
        \right)
        \left( 
            \sum_{I=1}^{m} \phi_{1I}
        \right)^{-1}
        \psi
        \\
        \vdots \\
        -\alpha
        \left( 
            \sum_{I=1}^{m} 
            C_{x,nI}
            \phi_{nI} 
        \right)
        \left( 
            \sum_{I=1}^{m} \phi_{nI}
        \right)^{-1}
        \psi
        \\
        %
        -\alpha
        \left( 
            \sum_{I=1}^{m} 
            C_{y,nI}
            \phi_{nI} 
        \right)
        \left( 
            \sum_{I=1}^{m} \phi_{nI}
        \right)^{-1}
        \psi
        \\
        %
        -\alpha
        \left( 
            \sum_{I=1}^{m} 
            C_{z,nI}
            \phi_{nI} 
        \right)
        \left( 
            \sum_{I=1}^{m} \phi_{nI}
        \right)^{-1}
        \psi
    \end{pmatrix}
    ,
\end{equation}
and taking $-\alpha \psi$ as a common factor
\begin{equation}
    \nabla
    \psi \left( \mat{r} \right)
    =
    -\alpha
    \begin{pmatrix}
        \left( 
            \sum_{I=1}^{m} 
            C_{x,1I}
            \phi_{1I} 
        \right)
        \left( 
            \sum_{I=1}^{m} \phi_{1I}
        \right)^{-1}
        \\
        %
        \left( 
            \sum_{I=1}^{m} 
            C_{y,1I}
            \phi_{1I} 
        \right)
        \left( 
            \sum_{I=1}^{m} \phi_{1I}
        \right)^{-1}
        \\
        %
        \left( 
            \sum_{I=1}^{m} 
            C_{z,1I}
            \phi_{1I} 
        \right)
        \left( 
            \sum_{I=1}^{m} \phi_{1I}
        \right)^{-1}
        \\
        \vdots \\
        \left( 
            \sum_{I=1}^{m} 
            C_{x,nI}
            \phi_{nI} 
        \right)
        \left( 
            \sum_{I=1}^{m} \phi_{nI}
        \right)^{-1}
        \\
        %
        \left( 
            \sum_{I=1}^{m} 
            C_{y,nI}
            \phi_{nI} 
        \right)
        \left( 
            \sum_{I=1}^{m} \phi_{nI}
        \right)^{-1}
        \\
        %
        \left( 
            \sum_{I=1}^{m} 
            C_{z,nI}
            \phi_{nI} 
        \right)
        \left( 
            \sum_{I=1}^{m} \phi_{nI}
        \right)^{-1}
    \end{pmatrix}
    \psi
    ,
\end{equation}

Therefore, the drift vector defined in \cref{eq:definition-drift-vector} is
given by
\begin{equation} \label{eq:final-drift-vector}
    \therefore
    \frac{
        \nabla \psi \left( \mat{r} \right)
    }{
        \psi \left( \mat{r} \right)
    }
    =
    -\alpha
    \begin{pmatrix}
        \left( 
            \sum_{I=1}^{m} 
            C_{x,1I}
            \phi_{1I} 
        \right)
        \left( 
            \sum_{I=1}^{m} \phi_{1I}
        \right)^{-1}
        \\
        %
        \left( 
            \sum_{I=1}^{m} 
            C_{y,1I}
            \phi_{1I} 
        \right)
        \left( 
            \sum_{I=1}^{m} \phi_{1I}
        \right)^{-1}
        \\
        %
        \left( 
            \sum_{I=1}^{m} 
            C_{z,1I}
            \phi_{1I} 
        \right)
        \left( 
            \sum_{I=1}^{m} \phi_{1I}
        \right)^{-1}
        \\
        \vdots \\
        \left( 
            \sum_{I=1}^{m} 
            C_{x,nI}
            \phi_{nI} 
        \right)
        \left( 
            \sum_{I=1}^{m} \phi_{nI}
        \right)^{-1}
        \\
        %
        \left( 
            \sum_{I=1}^{m} 
            C_{y,nI}
            \phi_{nI} 
        \right)
        \left( 
            \sum_{I=1}^{m} \phi_{nI}
        \right)^{-1}
        \\
        %
        \left( 
            \sum_{I=1}^{m} 
            C_{z,nI}
            \phi_{nI} 
        \right)
        \left( 
            \sum_{I=1}^{m} \phi_{nI}
        \right)^{-1}
    \end{pmatrix}
    ,
\end{equation}
where $C_{\lambda,iI}$ is given by \cref{eq:definition-CiI}.

% ------
\section{Laplacian \& kinetic energy}
% ------
The laplacian given in \cref{eq:definition-laplacian} is given by summing
over all $i$ electrons 
\begin{equation} \label{eq:laplacian-1}
    \laplacian \psi \left( \mat{r} \right)
    =
    \sum_{i=1}^{n} \partial_{x_i}^{2} \psi +
    \sum_{i=1}^{n} \partial_{y_i}^{2} \psi +
    \sum_{i=1}^{n} \partial_{z_i}^{2} \psi
    ,
\end{equation}
substituting \cref{eq:derivada-psi2-lambda}
into \cref{eq:laplacian-1} and taking $-\alpha \psi$ as a common factor
\begin{equation}
    \laplacian \psi \left( \mat{r} \right)
    =
    -\alpha
    \sum_{i=1}^{n} 
    \left[
        \frac{
            \sum_{I=1}^{m} 
            D_{x,iI}
            \phi_{iI}
        }{
            \sum_{I=1}^{m} \phi_{iI}
        }
        +
        \frac{
            \sum_{I=1}^{m} 
            D_{y,iI}
            \phi_{iI}
        }{
            \sum_{I=1}^{m} \phi_{iI}
        }
        +
        \frac{
            \sum_{I=1}^{m} 
            D_{z,iI}
            \phi_{iI}
        }{
            \sum_{I=1}^{m} \phi_{iI}
        }
    \right]
    \psi \left( \mat{r} \right)
    ,
\end{equation}
which can be rewritten as
\begin{equation} \label{eq:laplacian-2}
    \laplacian \psi \left( \mat{r} \right)
    =
    -\alpha
    \sum_{i=1}^{n} 
    \left[
        \frac{
            \sum_{I=1}^{m} 
            \left(  
                D_{x,iI}
                +
                D_{y,iI}
                +
                D_{z,iI}
            \right)
            \phi_{iI}
        }{
            \sum_{I=1}^{m} \phi_{iI}
        }
    \right]
    \psi \left( \mat{r} \right)
    .
\end{equation}

Now, recalling \cref{eq:coeficiente-DiI-CiI} 
\begin{align}
    \notag
    D_{iI}
    =
    D_{x,iI}
    +
    D_{y,iI}
    +
    D_{z,iI}
    &=
    \left[
        \frac{
            1
        }{
            \left| \mat{r}_{iI} \right|
        }
        -
        \frac{
            C_{x,iI}^{2}
            \left( 
                1 + \alpha \left| \mat{r}_{iI} \right|
            \right)
        }{
            \left| \mat{r}_{iI} \right|
        }
    \right]
    \\
    \notag
    \phantom{=}
    &+
    \left[
        \frac{
            1
        }{
            \left| \mat{r}_{iI} \right|
        }
        -
        \frac{
            C_{y,iI}^{2}
            \left( 
                1 + \alpha \left| \mat{r}_{iI} \right|
            \right)
        }{
            \left| \mat{r}_{iI} \right|
        }
    \right]
    \\
    \phantom{=}
    &+
    \left[
        \frac{
            1
        }{
            \left| \mat{r}_{iI} \right|
        }
        -
        \frac{
            C_{z,iI}^{2}
            \left( 
                1 + \alpha \left| \mat{r}_{iI} \right|
            \right)
        }{
            \left| \mat{r}_{iI} \right|
        }
    \right]
    ,
\end{align}
which can be simplified to
\begin{equation} \label{eq:DiI-1}
    D_{iI}
    =
    \frac{3}{\left| \mat{r}_{iI} \right|}
    -
    \frac{
        \left( 
            1 + \alpha \left| \mat{r}_{iI} \right|
        \right)
    }{
        \left| \mat{r}_{iI} \right|
    }
    \left( 
        C_{x,iI}^{2} + C_{y,iI}^{2} + C_{z,iI}^{2}
    \right)
    .
\end{equation}

Recalling the definition of $C_{\lambda,iI}$ in \cref{eq:definition-CiI}
\begin{equation}
    C_{x,iI}^{2} + C_{y,iI}^{2} + C_{z,iI}^{2}
    =
    \frac{\left( x_i - X_I \right)^{2}}{\left| \mat{r}_{iI} \right|^{2}} +
    \frac{\left( y_i - Y_I \right)^{2}}{\left| \mat{r}_{iI} \right|^{2}} +
    \frac{\left( z_i - Z_I \right)^{2}}{\left| \mat{r}_{iI} \right|^{2}}
    ,
\end{equation}
or, taking $1 / \left| \mat{r}_{iI} \right|^{2}$ as common factor
\begin{equation}
    C_{x,iI}^{2} + C_{y,iI}^{2} + C_{z,iI}^{2}
    =
    \frac{1}{\left| \mat{r}_{iI} \right|^{2}}
    \left[ 
        \left( x_i - X_I \right)^{2} +
        \left( y_i - Y_I \right)^{2} +
        \left( z_i - Z_I \right)^{2}
    \right]
    .
\end{equation}

By the definition given in \cref{eq:definition-mod-rR} 
\begin{equation}
    \left| \mat{r}_{iI} \right|^{2}
    =
    \left( x_i - X_I \right)^{2} +
    \left( y_i - Y_I \right)^{2} +
    \left( z_i - Z_I \right)^{2}
    ,
\end{equation}
it results in 
\begin{equation}
    C_{x,iI}^{2} + C_{y,iI}^{2} + C_{z,iI}^{2}
    =
    1
    .
\end{equation}

In \cref{eq:DiI-1}
\begin{equation} \label{eq:DiI}
    D_{iI}
    =
    \frac{3}{\left| \mat{r}_{iI} \right|}
    -
    \frac{
        \left( 
            1 + \alpha \left| \mat{r}_{iI} \right|
        \right)
    }{
        \left| \mat{r}_{iI} \right|
    }
    .
\end{equation}

Then, in \cref{eq:laplacian-2}
\begin{equation} \label{eq:final-laplacian}
    \laplacian \psi \left( \mat{r} \right)
    =
    -\alpha
    \sum_{i=1}^{n} 
    \left(
        \frac{
            \sum_{I=1}^{m} 
            D_{iI}
            \phi_{iI}
        }{
            \sum_{I=1}^{m} \phi_{iI}
        }
    \right)
    \psi \left( \mat{r} \right)
    .
\end{equation}

Therefore, the kinetic energy defined in \cref{eq:definition-kinetic-energy} is
given by 
\begin{equation}
    \therefore
    T_{\mathrm{L}} \left( \mat{r} \right)
    =
    -\frac{1}{2}
    \sum_{i=1}^{n} 
    \left(
        \frac{
            \sum_{I=1}^{m} 
            - \alpha
            D_{iI}
            \phi_{iI}
        }{
            \sum_{I=1}^{m} \phi_{iI}
        }
    \right)
    ,
\end{equation}
where $D_{iI}$ is defined in \cref{eq:DiI}.
